\documentclass{ComputationalAlgorithmsArticle}

\usepackage[dvips]{graphicx}
\usepackage{float}
\usepackage{subfigure}

\usepackage[dvips,
bookmarks,
bookmarksopen,
backref,
colorlinks,linkcolor={blue},citecolor={blue},urlcolor={blue},
]{hyperref}

\title{VTK/PCL File Conversions}

% 
% NOTE: This is the last number of the "handle" URL that 
% The Insight Journal assigns to your paper as part of the
% submission process. Please replace the number "1338" with
% the actual handle number that you get assigned.
%
\newcommand{\IJhandlerIDnumber}{3303}

% Increment the release number whenever significant changes are made.
% The author and/or editor can define 'significant' however they like.
\release{0.00}

% At minimum, give your name and an email address.  You can include a
% snail-mail address if you like.

\author{David Doria and Jaudiel Velez-robles}
\authoraddress{Army Research Laboratory, Aberdeen MD and Army Geospatial Agency}


\begin{document}

\IJhandlefooter{\IJhandlerIDnumber}


\ifpdf
\else
   %
   % Commands for including Graphics when using latex
   % 
   \DeclareGraphicsExtensions{.eps,.jpg,.gif,.tiff,.bmp,.png}
   \DeclareGraphicsRule{.jpg}{eps}{.jpg.bb}{`convert #1 eps:-}
   \DeclareGraphicsRule{.gif}{eps}{.gif.bb}{`convert #1 eps:-}
   \DeclareGraphicsRule{.tiff}{eps}{.tiff.bb}{`convert #1 eps:-}
   \DeclareGraphicsRule{.bmp}{eps}{.bmp.bb}{`convert #1 eps:-}
   \DeclareGraphicsRule{.png}{eps}{.png.bb}{`convert #1 eps:-}
\fi


\maketitle


\ifhtml
\chapter*{Front Matter\label{front}}
\fi

\begin{abstract}
\noindent

This document presents a set of simple conversions between the Point Cloud Library's PCD file format and the Visualization Toolkit's VTP file format.

The code is available here:
\begin{verbatim}
https://github.com/daviddoria/VTK_PCL_Conversions 
\end{verbatim}


\end{abstract}

\IJhandlenote{\IJhandlerIDnumber}

\tableofcontents
%%%%%%%%%%%%%%%%%%%%
\section{Introduction}
This document presents a set of simple conversions between the Point Cloud Library's (PCL) Point Cloud Data (PCD) file format and the Visualization Toolkit's (VTK) PolyData (VTP) file format. In PCL, point clouds are represented as a collection of coordinates along with optional additional attributes. PCL provides some standard sets of attributes in the form of structs that include PointXYZ for coordinate-only points, PointXYZRGB for colored points, and PointXYZRGBNormal for colored points with associated normal vectors. In VTK, these different data attributes are not explicitly grouped, but rather one can add any number of any type of attribute to the base coordinate data. That is, for simple coordinate-only points, a vtkPolyData objects only containing a vtkPoints is used. For colored points, a vtkPoints object with an associated vtkUnsignedCharArray is used. For colored points with associated normal vectors, a vtkPoints object with associated vtkUnsignedCharArray for the colors and a vtkFloatArray for the normals is used. These equivalent representations are summarized in the table below.

\begin{center}
 
\begin{tabular}{ l | l }
  PCL & VTK\\
  \hline
  PointXYZ & vtkPoints\\
  PointXYZRGB & vtkPoints + vtkUnsignedCharArray \\
  PointXYZRGBNormal & vtkPoints + vtkUnsignedCharArray + vtkFloatArray\\
  PointNormal & vtkPoints + vtkFloatArray\\
\end{tabular}

\end{center}


%%%%%%%%%%%%%%%%%%%%
\section{VTK to PCD}
We provide a simple, templated interface to the conversions function.
\begin{verbatim}
template <typename PointT> 
void VTKtoPCL(vtkPolyData* pdata, typename pcl::PointCloud<PointT>::Ptr cloud)
\end{verbatim}

The generic template takes any vtkPolyData point-only pcl::PointCloud. This generic template handles the case of PointXYZ. Specializations handle the cases of PointXYZRGB, PointXYZRGBNormal, and PointNormal by additionally adding the extra attributes to the PointT struct.

\subsection{Code Snippet}

\begin{verbatim}
  // Read the VTP file
  vtkSmartPointer<vtkXMLPolyDataReader> reader = vtkSmartPointer<vtkXMLPolyDataReader>::New();
  reader->SetFileName(inputFileName.c_str());
  reader->Update();
  
  // Create the output point cloud
  pcl::PointCloud<pcl::PointXYZ>::Ptr cloud(new pcl::PointCloud<pcl::PointXYZ>);

  // Convert the data
  VTKtoPCL<pcl::PointXYZ>(reader->GetOutput(), cloud);
 
  // Save the output
  pcl::io::savePCDFileASCII (outputFileName.c_str(), *cloud);
\end{verbatim}

%%%%%%%%%%%%%%%%%%%%
\section{PCD to VTK}
We provide a simple, templated interface to the conversions function.
\begin{verbatim}
template <typename PointT> 
void PCLtoVTK(typename pcl::PointCloud<PointT>::Ptr cloud, vtkPolyData* pdata)
\end{verbatim}

The generic template takes any PCL point struct that has .x, .y, and .z members available and converts it to a point-only vtkPolyData. This generic template handles the case of PointXYZ. Specializations handle the cases of PointXYZRGB, PointXYZRGBNormal, and PointNormal by additionally adding the extra attributes to the vtkPolyData.

\subsection{Code Snippet}

\begin{verbatim}
  // Load the PCL PCD file
  pcl::PointCloud<pcl::PointXYZ>::Ptr cloud (new pcl::PointCloud<pcl::PointXYZ>);
  pcl::io::loadPCDFile<pcl::PointXYZ> (inputFileName.c_str(), *cloud);

  // Create a polydata object.
  vtkSmartPointer<vtkPolyData> polydata = vtkSmartPointer<vtkPolyData>::New();

  // Convert the PCL data to VTK data
  PCLtoVTK<pcl::PointXYZ>(cloud, polydata);

  // Write the output
  vtkSmartPointer<vtkXMLPolyDataWriter> writer =
    vtkSmartPointer<vtkXMLPolyDataWriter>::New();
  writer->SetFileName(outputFileName.c_str());
  writer->SetInputConnection(polydata->GetProducerPort());
  writer->Write();
  
\end{verbatim}


\end{document}
